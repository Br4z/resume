\documentclass[11pt,a4paper,sans]{moderncv}
\moderncvstyle{classic}
\moderncvcolor{black}
\usepackage[scale=0.75]{geometry}
\usepackage[english]{babel}
\usepackage{csquotes}
\usepackage{biblatex}
\addbibresource{publications.bib}

\ifxetexorluatex
	\usepackage{fontspec}
	\usepackage{unicode-math}
	\defaultfontfeatures{Ligatures=TeX}
	\setmainfont{Latin Modern Roman}
	\setsansfont{Latin Modern Sans}
	\setmonofont{Latin Modern Mono}
	\setmathfont{Latin Modern Math}
\else
	\usepackage[T1]{fontenc}
	\usepackage{lmodern}
\fi

% ------------------------------- PERSONAL DATA ------------------------------ %

\name{Brandon}{Calderón Prieto}
\title{Estudiante de Ingeniería de Sistemas}
%\born{26 Diciembre 2003}
\address{}{Cali}{Colombia}
\phone[mobile]{(+57) 301 5088603}
\email{bcalderonprieto@gmail.com}
%\homepage{}

\social[github]{br4z}

\photo[64pt][0.4pt]{picture}

\begin{document}

\makecvtitle

\section{Perfil Profesional}

Estudiante de Ingeniería de Sistemas con gran motivación y experiencia práctica en herramientas modernas como Docker, Git y Python. Me apasiona crear soluciones eficientes y resolver problemas complejos mediante la tecnología. Deseo contribuir con mis capacidades de colaboración y mentoría técnica en un rol profesional en el campo de la ingeniería de software o infraestructura tecnológica.

\section{Educación}

\cventry{2010--2020}{Bachillerato con especialidad en humanidades}{Institución Educativa Santa Fe}{Cali}{}{}
\cventry{2021--actualidad}{Ingeniería de Sistemas}{Universidad del Valle}{Cali}{pregrado}{}

\section{Experiencia}

\cventry{2023--2025}{Monitor Socio-Académico}{Estrategia ASES - Universidad del Valle}{Cali}{}{
\begin{itemize}
	\item Fomenté una cultura de curiosidad y superación, motivando a estudiantes de primeros semestres a explorar tecnologías y lenguajes de programación más allá del currículo académico.
	\item Diseñé y lideré talleres prácticos sobre herramientas de desarrollo modernas (Git, Docker) y buenas prácticas, acelerando su curva de aprendizaje técnico.
	\item Guié a los estudiantes en la resolución de desafíos académicos complejos, fortaleciendo no solo su conocimiento técnico sino también su confianza y autonomía.
\end{itemize}}

\section{Lenguajes}

\cvitemwithcomment{Español}{nativo}{}

\cvitemwithcomment{Inglés}{B2}{certificado por la Universidad del Valle}

\section{Habilidades}

\cvskilllegend[0.5em]
	[Básico]
	[Intermedio]
	[Avanzado]
	[Experto]
	[Maestro]
	{}
\cvskillhead[-0.1em]
	[Nivel]
	[Habilidad]
	[Años]
	[Comentario]
\cvskillentry*{Lenguajes:}{3}{Python}{2}{Desarrollo backend}
\cvskillentry{}{3}{JavaScript}{2}{}
\cvskillentry{}{2}{HTML + CSS}{2}{Desarrollo Web básico}
\cvskillentry*{SO:}{3}{Linux}{3}{Administración y personalización}
\cvskillentry*{Herramientas:}{3}{Git y GitHub}{4}{}
\cvskillentry{}{3}{Docker}{2}{}

\section{Competencias}

\cvlistitem{Trabajo en equipo y colaboración proactiva.}
\cvlistitem{Comunicación técnica y mentoría.}
\cvlistitem{Resolución de problemas en grupo.}
\cvlistitem{Liderazgo y motivación de equipos.}

\section{Intereses}

\cvlistitem{Cultura del software libre y de código abierto (FOSS).}
\cvlistitem{Exploración y personalización de diversos entornos de escritorio en Linux.}
\cvlistitem{Automatización de tareas y flujos de trabajo mediante scripting.}

% \section{Publicaciones}

% \printbibliography

% \section{Referencias}

\end{document}
